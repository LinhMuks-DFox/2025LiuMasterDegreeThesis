\section{Problem Setting}
\label{sec:problem_setting}

We address the challenge of implementing the AEC framework using Blinkies. This section formalizes the AEC pipeline and defines the core research problem.

\subsection{Formalization of AEC Framework using Blinkies}
The AEC procedure using $N$ Blinkies consists of three sequential stages: on-device embedding ($\mathrm{Embed}(\cdot)$), optical signal transmission ($\mathrm{Transmit}(\cdot)$), and downstream classification ($\mathrm{Classify}(\cdot)$). This pipeline is expressed as:
\begin{align}
    \boldsymbol{z}_i  & = \mathrm{Embed}(\boldsymbol{x}_i),\label{eq:ps_embedding}                               \\
    \boldsymbol{z'}_i & = \mathrm{Transmit}(\boldsymbol{z}_i), \label{eq:ps_channel}                             \\
    \hat{y}           & = \mathrm{Classify}(\boldsymbol{z'}_1, \cdots, \boldsymbol{z}_N). \label{eq:ps_classify}
\end{align}
\begin{figure*}[ht]
    \centering
    \includegraphics[width=\textwidth]{sections/pdf-png-ref/AEC-Diagram.pdf}
    \caption{
        Diagram of AEC framework via blinky array.
    }
    \label{fig:aec_diagram}
\end{figure*}
As illustrated in Fig.~\ref{fig:proposed_system_diagram},
the $i$-th Blinky first acquires an acoustic signal $\boldsymbol{x}_i$.
The embedding function $\mathrm{Embed}(\cdot)$ on the Blinky maps the captured signal to a feature vector $\boldsymbol{z}_i \in \mathbb{R}^{L}$, which subsequently modulates the device's LEDs.
The resulting light signal propagates through the environment and is captured by a central camera;
the composite process is abstracted by the function $\mathrm{Transmit}(\cdot)$.
Finally, the classifier $\mathrm{Classify}(\cdot)$ operates on the collection of received signals $\{\boldsymbol{z'}_i\}$ to infer the acoustic event label $\hat{y}$.

Within the transmission channel, Blinkies' LED signals are perturbed by ambient illumination and noise before being temporally resampled by the camera.
Accordingly, we model the channel as the composition of a distortion operator followed by camera resampling, i.e., $\mathrm{Transmit}(\cdot) = (\mathrm{Resample} \circ \mathrm{Distort}(\cdot))$.
The distortion is parameterized as follows:
\begin{align}
    \mathrm{Distort}(\boldsymbol{z}) & = a\boldsymbol{z} + b\boldsymbol{1} + \boldsymbol{\epsilon}, \quad \boldsymbol{\epsilon} \sim \mathcal{N}(0, \sigma^2\mathbf{I}), \label{eq:ps_distort}
\end{align}
where $a$ denotes the attenuation coefficient---inversely proportional to the squared distance between the camera and a Blinky---and $b$ captures the bias introduced by ambient light,
whereas $\boldsymbol{\epsilon}$ represents additive white Gaussian noise with covariance $\sigma^2 \boldsymbol{I}$.
The vector $\boldsymbol{1}$ is the $L$-dimensional all-ones vector
and the matrix $\boldsymbol{I}$ is the identity matrix with a size of $L \times L$.
The resampling operator $\mathrm{Resample}: \mathbb{R}^{L} \rightarrow \mathbb{R}^{N_{\text{LED}} \times \lfloor F_s \times T \rfloor}$ encapsulates the camera's temporal sampling, governed by its frame rate $F_s$, the number of LEDs per device $N_{\text{LED}}$, and the audio segment duration $T$.
To respect the sensor's dynamic range,
resampled values are clipped to the interval $[0, 1]$.

\subsection{Challenge: Designing Embedding Function}
Within this framework, the central challenge is to design an embedding function $\mathrm{Embed}(\cdot)$ that retains as much acoustic information as possible despite stringent bandwidth constraints and pronounced channel noise.
To date, two principal strategies have been explored.

\paragraph{Sound Power}
\begin{figure*}[ht]
    \centering
    \includegraphics[width=\textwidth]{sections/pdf-png-ref/sound_power_approach.pdf}
    \caption{
        Diagram of sound power approach.
    }
    \label{fig:sound_power_diagram}
\end{figure*}

As shown in~\ref{fig:sound_power_diagram}, the simplest method defines $\mathrm{Embed}(\cdot)$ by mapping the raw waveform $\boldsymbol{x}(t)$ to its power: $\boldsymbol{z} = (x(1)^2, x(2)^2, \cdots, x(T)^2)^\top$~\cite{scheibler_blinkies_2020}.
Although computationally negligible, the downsampling process in the camera discards most of the semantic information (e.g., timbre), resulting in insufficient performance on AEC tasks.

\paragraph{End-to-End Training Approach}
\begin{figure*}[ht]
    \centering
    \includegraphics[width=\textwidth]{sections/pdf-png-ref/end_to_end.pdf}
    \caption{
        Diagram of End-to-End approach proposed by Kinoshita et al.
    }
    \label{fig:end_to_end}
\end{figure*}
As shown in~\ref{fig:end_to_end}, to convey more discriminative features, Kinoshita et al.~\cite{kinoshita_end--end_2021} proposed using a DNN encoder for $\mathrm{Embed}(\cdot)$ and optimized both $\mathrm{Embed}(\cdot)$ and $\mathrm{Classify}(\cdot)$ (Eqs.~\ref{eq:ps_embedding} and \ref{eq:ps_classify}) in an end-to-end fashion.
However, this strategy is impractical in real-world deployments:
the transmit channel (Eq.~\ref{eq:ps_channel}) constitutes an unknown, non-differentiable physical process, thereby blocking gradient flow back to the on-device encoder.

Both approaches therefore exhibit critical shortcomings: the sound-power approach is too rudimentary for high-accuracy classification, whereas the end-to-end training approach is physically unrealizable. Consequently, a practical solution is required---one that permits the transmission of expressive, learned features without necessitating differentiation through the physical channel.
