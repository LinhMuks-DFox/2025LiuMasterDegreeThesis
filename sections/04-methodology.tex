\section{Methodology}
\label{sec:methodology}
\begin{figure*}[ht]
    \centering
    \includegraphics[width=\textwidth]{sections/pdf-png-ref/proposal_and_conventional.pdf}
    \caption{Overview of the proposed framework.
        \textbf{Pre-training phase} (bottom): An autoencoder (AE) is pre-trained to learn noise-robust representations; Gaussian noise ($\boldsymbol{\epsilon}$) is injected into the latent vector ($\boldsymbol{z}$) and the decoder ($D_{\theta}$) reconstructs the original log-Mel spectrogram ($\boldsymbol{X}$).
        \textbf{AEC phase} (top): At inference, the frozen encoder ($E_{\phi}^{*}$) produces a compact latent representation that modulates four LEDs. A central camera records the signal as \emph{temporal LED intensity variations}, and the downstream classifier predicts the event label ($\hat{\boldsymbol{y}}$).}
    \label{fig:proposed_system_diagram}
\end{figure*}
To overcome the limitations described in the previous section, we propose a practical solution that decouples the feature embedding from the non-differentiable physical channel.
Our approach, illustrated in Fig.~\ref{fig:proposed_system_diagram}, realizes the embedding function $\mathrm{Embed}(\cdot)$ via the encoder of a pre-trained autoencoder (AE), denoted as $E_{{\phi}^{*}}(\cdot)$.

The procedure begins by transforming the raw input waveform $\boldsymbol{x}$ into a log-Mel spectrogram $\boldsymbol{X}=\mathrm{Logmel}(\boldsymbol{x})$,
a widely adopted time-frequency representation of audio.
The embedding function then maps the waveform to a latent vector $\boldsymbol{z}$ through the log-Mel representation:
\begin{equation}
    \boldsymbol{z} = \mathrm{Embed}(\boldsymbol{x}) = E_{{\phi}^{*}}(\boldsymbol{X}).
    \label{eq:embed_ae}
\end{equation}
By pre-training the encoder,
we obtain a compact yet discriminative representation that is suited for transmission,
thereby obviating the need for end-to-end optimization through the physical channel.
This section details the AE's learning objective, its network architecture, pre-training procedure, and training procedure for the downstream classifier.
\footnote{All training scripts, including Autoencoder pre-training and downstream evaluation, are available at \url{https://github.com/ykinolab-tokai/multi-ae-ace}.}

\subsection{Learning Objective}
The primary objective of an AE is to distill a compact representation from a large, unlabeled audio dataset (e.g. Google AudioSet \cite{gemmeke_audio_2017}) by means of a self-supervised reconstruction task.
A standard AE minimizes the reconstruction error between an input log-Mel spectrogram $\boldsymbol{X}$ and its reconstruction $\hat{\boldsymbol{X}}$ via
\begin{equation}
    \mathcal{L}_{\text{rec}} = \mathbb{E}\left[ ||\boldsymbol{X} - D_\theta(E_\phi(\boldsymbol{X}))||^2_F \right],
    \label{eq:recon_loss}
\end{equation}
where $E_\phi(\cdot)$ and $D_\theta(\cdot)$ represent the encoder and decoder, respectively;
the optimized parameters after pre-training are $(\phi^*, \theta^*)$.

However, this objective does not consider the perturbation of the latent vector $E_\phi(\boldsymbol{X})$ by channel noise.
To address this issue, we adopt a noise-robust training strategy:
Gaussian noise $\boldsymbol{\epsilon}$ is injected into the latent vectors during training to mimic channel distortion and compel the model to produce a more resilient representation.
The revised objective is
\begin{equation}
    \mathcal{L}_{\text{rec-robust}} = \mathbb{E}\left[ ||\boldsymbol{X} - D_\theta(E_\phi(\boldsymbol{X}) + \boldsymbol{\epsilon})||^2_F \right], \quad \boldsymbol{\epsilon} \sim \mathcal{N}(0, \sigma^2\mathbf{I}),
    \label{eq:recon_loss_robust}
\end{equation}
thereby encouraging $E_\phi(\cdot)$ to produce channel-invariant and discriminative features.
\subsection{AE Architecture}
\label{subsec:ae_architecture_rationale}
\begin{table}[t]
    \centering
    \caption{
        Autoencoder architecture.
        Output shapes are listed as $\mathrm{frequency} \ F \times \mathrm{time} \ T$.
        Self-Attention block consists of multi-head attention layer with two heads, layer normalization, feed-forward network (FFN) that expands feature dimension from 40 to 128 and then projects it back, and layer normalization with residuals.
        $k$, $s$, and $c$ denote kernel size, stride, and channel,
        respectively.
    }
    \label{tab:ae_architecture}
    \begin{tabular}{l c}
        \toprule
        \textbf{Layer}                                & \textbf{Output Shape} ($F \times T$) \\
        \midrule
        \textit{Input: Log-Mel Spectrogram}           & [80, 501]                            \\
        \midrule
        \multicolumn{2}{l}{\textbf{Encoder} ($E_\phi$)}                                      \\
        Conv1d (c=40, k=5, s=1), ReLU                 & [40, 497]                            \\
        Self-Attention (embed=40, heads=2)$^*$        & [40, 497]                            \\
        ReLU                                          & [40, 497]                            \\
        Conv1d (c=20, k=7, s=2), ReLU                 & [20, 246]                            \\
        Conv1d (c=20, k=11, s=3), ReLU                & [20, 79]                             \\
        Flatten                                       & [1, 1580]                            \\
        Linear (in\_features=1580, out\_features=300) & [1, 300]                             \\
        \midrule
        \multicolumn{2}{l}{\textbf{Decoder} ($D_\theta$)}                                    \\
        Linear (in\_features=300, out\_features=1580) & [1, 1580]                            \\
        Reshape                                       & [20, 79]                             \\
        ConvTranspose1d (c=20, k=11, s=3), ReLU       & [20, 245]                            \\
        ConvTranspose1d (c=40, k=7, s=2), ReLU        & [40, 495]                            \\
        ConvTranspose1d (c=80, k=5, s=1)              & [80, 499]                            \\
        Interpolate (target\_time\_dim=501)           & [80, 501]                            \\
        \midrule
        \textit{Output: Reconstructed Log-Mel}        & [80, 501]                            \\
        \bottomrule
    \end{tabular}
\end{table}

The Encoder architecture, summarized in Table~\ref{tab:ae_architecture}, is a hybrid design that combines convolutional neural networks (CNNs) and self-attention.
A 1-D convolutional layer first extracts local temporal patterns, after which a multi-head self-attention layer captures long-range dependencies, achieving a global receptive field without excessive network depth.
A series of strided convolutional layers performs temporal downsampling before projecting the signal into an $L$-dimensional latent space.
We choose $L$ to match the physical channel capacity, i.e.,
$L = 15 \, \mathrm{symbol/s} \times 4 \, \mathrm{LEDs} \times 5 \mathrm{s} = 300$, where $15$\,Hz is the per-LED Nyquist rate imposed by the 30\,FPS camera.
Unlike audio transformer models~\cite{gong2021astaudiospectrogramtransformer}, which prioritize accuracy without reducing bitrate, our AE explicitly compresses the representation to meet the optical-channel capacity.

A crucial deployment constraint is the encoder's computational footprint on resource-constrained hardware.
Our encoder contains 517,508 trainable parameters,
occupying roughly 2.04\,MB of memory,
and its total inference-time footprint is about 3.40\,MB.
This is comfortably within the capacity of modern edge platforms such as the Raspberry Pi 4.

\subsection{Pre-training}
\label{subsec:ae_preprocessing}
We aim to infer a single acoustic-event label $y$ from a set of 5-second audio waveform $\{ \boldsymbol{x}_i \}$ sampled at 16\,kHz.
Each 10-second audio clip from the Google AudioSet~\cite{gemmeke_audio_2017} is first down-mixed to monaural and resampled to 16\,kHz.
For data augmentation, a 5-second segment is then randomly cropped from every clip.
A log-Mel spectrogram $\boldsymbol{X}$ is then computed from this segment, using a Hann window with a length of 400, a 160-sample hop length, and 80 Mel filters.

The AE is pre-trained for 300 epochs on the balanced subset of AudioSet with the Adam optimizer~\cite{kingma_adam_2017}, an initial learning rate of $1 \times 10^{-3}$ and a batch size of 1280.
A \texttt{ReduceLROnPlateau} learning rate scheduler lowers the learning rate by a factor of 0.1 whenever the validation loss fails to improve for 10 consecutive epochs.

\subsection{Training Downstream Classifier}
\label{subsec:classifier_training}
To train the downstream classifier that predicts acoustic event labels
from the transmitted data, we first record the Blinky signals corresponding to each input audio instance in the training set
with the central camera, thereby obtaining a set of latent vectors $\{\boldsymbol{z}'_i\}$ (see Eqs.~\ref{eq:ps_embedding} and \ref{eq:ps_channel}).
The classifier is subsequently trained to infer the target labels $y$
from these latent representations,
whereas the parameters of the pre-trained encoder $E_{{\phi}^{*}}$ are kept fixed throughout this phase.